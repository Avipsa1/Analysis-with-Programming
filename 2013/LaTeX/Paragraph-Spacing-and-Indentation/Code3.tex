\documentclass[12pt]{article}
\title{History of Statistics}
\author{Al-Ahmadgaid B. Asaad}
 
%\usepackage{setspace}
%\doublespacing
\setlength{\parskip}{0.3cm}
 
 
\begin{document}
\maketitle
{\baselineskip = 2.0\baselineskip
Statistical methods date back at least to the 5th century BC.
The earliest known writing on statistics appears in a 9th-century
book entitled Manuscript on Deciphering Cryptographic Messages,
written by Al-Kindi. In this book, Al-Kindi provides a detailed
description of how to use statistics and frequency analysis to
decipher encrypted messages. This was the birth of both statistics
and cryptanalysis, according to the Saudi engineer Ibrahim Al-Kadi.
 
}
 
The Nuova Cronica, a 14th-century history of Florence by the
Florentine banker and official Giovanni Villani, includes much
statistical information on population, ordinances, commerce,
education, and religious facilities, and has been described as
the first introduction of statistics as a positive element in history.
 
Some scholars pinpoint the origin of statistics to 1663, with
the publication of Natural and Political Observations upon the
Bills of Mortality by John Graunt. Early applications of statistical
thinking revolved around the needs of states to base policy on
demographic and economic data, hence its stat- etymology. The scope
of the discipline of statistics broadened in the early 19th century
to include the collection and analysis of data in general. Today,
statistics is widely employed in government, business, and natural
and social sciences.
 
Its mathematical foundations were laid in the 17th century with the
development of the probability theory by Blaise Pascal and Pierre
de Fermat. Probability theory arose from the study of games of chance.
The method of least squares was first described by Adrien-Marie Legendre
in 1805. The use of modern computers has expedited large-scale statistical
computation, and has also made possible new methods that are impractical
to perform manually.
\end{document}
